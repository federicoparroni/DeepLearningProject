\section{Experiments}
In this section you validate your method showing the experiments that you performed. The experiments will vary depending on the project, but you might compare with previously published methods, perform an ablation study to determine the impact of various components of your system, experiment with different hyperparameters or architectural choices, use visualization techniques to gain insight into how your model works, discuss common failure modes of your model, etc. You should include graphs, tables, or other figures to illustrate your experimental results. Divide in subsections or paragraphs to help the reader navigate in your paper.

\paragraph{Datasets.}
Describe the data you are working with for your project. Usually you need to explain what type of data is it, how much data are you working with and if you applied any pre-processing, filtering, or other special treatment to use it. Remember that you have to cite each dataset you used in your project if it has been published from someone else. Instead, if you collected it by yourself you have to describe accurately how you gathered (and labeled) your data. 

\paragraph{Experiments setup.}
Here you describe all the architectural choices of your model, the hyper-parameters of your model, \eg optimizer, learning rate, momentum, batch size and if you cross-validate on them. 

\paragraph{Results and discussion.}
Discuss your results and compare with other methods. You can also perform an \emph{ablation study} on your model switching on and off some components to understand their contributions.
