\section{Related work}
%In this section you should discuss published work that relates to your project. This is expected to be full of references, meaning that you have read the existing literature and you know what you are working on very well. This is not just a list or works, but you are not supposed to cluster papers that use similar approaches and compare them each other using very short sentences. I strongly suggest to check \cite{steinhardt, lipton} blog posts for some good practices in writing a paper. A quote that I like very much is \emph{``Research is spending 6 hours reading 35 papers, so you can write one sentence containing 2 references''} \cite{twit:ref}. Keep it in mind while you are writing!
%------------------------------------------------------------------------
Our work is based on training a deep convolutional neural network, using a purely data driven approach, to extract features from couple of images than use it as input to a fully connected network to say wheter two images representing faces are of the same person or not.
Similar work has been carried out some using directly deep convolutional neural network to give a response other using it as a support for the extraction of features used as input for SVMs.
\\
The most relevant are reported below.
\\
\par
Schroff \textit{et al.} [] use a deep convolutional neural network to generate an embedding $f(x)$, from an image $x$ into a feature space $R^d$, such that the squared distance between all faces of the same identity is small, whereas the squared distance between a pair of face images from different identities is large. To achieve that has been used the so called 'triplet loss' that allows the faces for one identity to live on a manifold, while still enforcing the distance of two different faces to be maximum.
The work presented does not directly solve the problem of the face recognition but create a proper enviroment in which a simple K-NN classifier can solve the task.
\\
\par
Aleju \cite{aleju} The network presented uses two branches, one per image. Each branch applies a few convolutions and ends in a fully connected layer. The outputs of both branches are then merged and further processed by another fully connected layer, before making the final yes-no-decision (whether both images show the same person).
\\
\par
\cit{Taigman}{taigman} propose a multi-stage approach that aligns faces to a general 3D shape model. A multi-class net-work is trained to perform the face recognition task on over four thousand identities. The authors also experimented with a so called Siamese network where they directly opti-
mize the $L1$-distance between two face features. Their best performance on LFW (97.35\%) obtained from an ensemble of three networks using different alignments and color channels. The predicted distances (non-linear SVM predictions) of those networks are combined using a non-linear SVM.
\\
\par
\cit{Zhenyao}{zhenyao} employ a deep network  to “warp” faces into a canonical frontal view and then learn CNN that classifies each face as belonging to a known identity. For face verification, PCA on the network output in conjunction with an ensemble of SVMs is used.
