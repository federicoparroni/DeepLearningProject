\section{Introduction}
This is the first section of your
 paper and you should contextualize the problem you are working on, why it is important, and give an overview of your results. It is useful to list all the contributions of your work very clearly, so that the reviewer can easily understand the value of your paper. This is helpful to guide the reader through the paper. If your project consists in replicating and/or extending another paper, then you should be very clear about it explaining what you did and how you proceeded.
For instance, the main contribution of this paper are: 
\begin{itemize}
    \item A brief tutorial on how to write a paper/technical report.
    \item Specific good/bad examples of writing.
    \item An example of structure that is suggested to follow for a technical report and can be reorganized as you need.
\end{itemize}
Each researcher has its own style and preferences when it comes to write a paper, so feel free to modify the structure as it pleases you. 
This tutorial is especially directed to master students who might need more help to organize their reports.

\subsection{Language}

All manuscripts must be in English. Do not use contraction forms, \eg \emph{doesn't} $\xrightarrow{}$ \emph{does not}, \emph{isn't} $\xrightarrow{}$ \emph{is not}. 

\subsection{Paper length}
Papers, excluding the references section, must be between \textbf{4-6} pages, bust must be no longer than \textbf{six pages}. The maximum number of pages is strict. Overlength papers will simply not be graded or graded negatively. Note that this \LaTeX\ guide already sets figure captions and references in a smaller font.

The references section will not be included in the page count, and there is no limit on the length of the references section. For example, a paper of six pages with two pages of references would have a total length of 8 pages, so it is fine.


%------------------------------------------------------------------------