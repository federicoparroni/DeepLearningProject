\section{Proposed approach}
%This is the core of your paper, where you describe the details of the proposed method for solving the problem that you set up in the introduction. This is the most important section. It has to be clear why the chosen approach is the right thing to do with respect to the possible alternatives. The explanation of the method has to be readable and understandable and it should not raise obvious questions from the reader. 
%
%You can divide the section in paragraphs or subsections that can be useful for the presentation of your method. Usually at this point you may want to place equations, figures and tables to clarify what you are explaining.

To accomplish our work we used a CNN. This kind of model is now a days the most widely used for Image Recognition tasks. 

\cite{einstein}
%-------------------------------------------------------------------------
%\subsection{Mathematics}
%Please number all of your sections and displayed equations.  It is
%important for readers to be able to refer to any particular equation.  Just
%because you didn't refer to it in the text does not mean some future reader
%might not need to refer to it.  It is cumbersome to have to use
%circumlocutions like ``the equation second from the top of page 3 column
%1''. 
%%-------------------------------------------------------------------------
%\subsection{Footnotes}
%Please use footnotes\footnote {This is what a footnote looks like.  It
%often distracts the reader from the main flow of the argument.} sparingly.
%Indeed, try to avoid footnotes altogether and include necessary peripheral
%observations in
%the text (within parentheses, if you prefer, as in this sentence).  If you
%wish to use a footnote, place it at the bottom of the column on the page on
%which it is referenced. Use Times 8-point type, single-spaced.
%%-------------------------------------------------------------------------
%\subsection{References}
%List and number all bibliographical references in 9-point Times,
%single-spaced, at the end of your paper. When referenced in the text,
%enclose the citation number in square brackets, for
%example~\cite{Authors14}.  Where appropriate, include the name(s) of
%editors of referenced books.
%%-------------------------------------------------------------------------
%\subsection{Illustrations, graphs, and photographs}
%All graphics should be centered.  Please ensure that any point you wish to
%make is resolvable in a printed copy of the paper.  Resize fonts in figures
%to match the font in the body text, and choose line widths which render
%effectively in print.  Many readers (and reviewers), even of an electronic
%copy, will choose to print your paper in order to read it.  You cannot
%insist that they do otherwise, and therefore must not assume that they can
%zoom in to see tiny details on a graphic.
%
%When placing figures in \LaTeX, it's almost always best to use
%\verb+\includegraphics+, and to specify the  figure width as a multiple of
%the line width as in the example below
%{\small\begin{verbatim}
%   \usepackage[dvips]{graphicx} ...
%   \includegraphics[width=0.8\linewidth]
%                   {myfile.eps}
%\end{verbatim}
%}

%------------------------------------------------------------------------
